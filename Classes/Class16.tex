\chapter{Class UNK - Monday, October 23\ts{rd}, 2017}
\section{\S 2.8 Subspaces}

\begin{imp:defn}{Subspace}{} A subspace of rton is any set H in  that has three properties:
\begin{itemize}
  \item  is in H
  \item For each u, v, in H u+v is also in H (Closure under addition)
  \item For each u in H and scalar C, cu is in H (Closure under scalar multiplication)
\end{itemize}
\end{imp:defn}
\begin{ex}
%image
fdsaf
Is H a subspace of rto2?
\begin{itemize}
\item is in h \checkmark
\item u+v is not in H for all u,v in H
\item Not closed under scalar multiplication either (-u not in H).
\end{itemize}
H is not a subspace
\end{ex}
\begin{ex}
%image
\begin{itemize}
    \item is in H \checkmark
    \item closed under addition \checkmark
    \item closed under scalar multiplication \checkmark
\end{itemize}
H is a subspace of rto2
\end{ex}
\begin{ex}
Which of the following are subspaces of $\mathbb{R}^{2}$?

\begin{itemize}
    \item %image\\
    not a subspace o is not in h
    \item %imgae\\
    $\mathbb{R}^2$ is a subspace of $\mathbb{R}^2$
    \item %image\\
    $\mathbb{R}^2$ is a subspace of $\mathbb{R}^2$
    \item %image\\
    \begin{itemize}
        \item o is in H \checkmark
        \item Closed under scalar multiplicaiton \checkmark
        \item Not closed unfer addition x u+(-v) in in H.
    \end{itemize}
    not a subspace of $\mathbb{R}^2$
    \item %image% 
    stuff
\end{itemize}
\end{ex}
So what are do subspaces of $\mathbb{R}^2$ look like?
\begin{itemize}
    \item They are copies of $\mathbb{R}^0$, $\mathbb{R}^1$, $\mathbb{R}^2$... $\mathbb{R}^n$ that contain the zero vector.
\end{itemize}
\begin{ex}
In $\mathbb{R}^3$, possible subspaces are:
\begin{itemize}
    \item Zero Subspaces
    \item Lines
    \item Planes
\end{itemize}
\end{ex}
\begin{ex}
If H=span{$v_1$,$v_2$} ({$av_1$,$bv_2$} for any a,b), then H is a subspace of $\mathbb{R}^n$
\begin{itemize}
\item o is in H \checkmark
\begin{itemize}
    \item since o*$v_1$ +o*$v_2$=o
\end{itemize}
\item closed under addition \checkmark
\begin{itemize}
    \item $u=a_1v_1+b_1v_2$
    \item $w=a_2v_1+b_2v_2$
    \item $u+w=a_1v_1+b_1v_2+a_2v_1+b_2v_2$
    \item $u+w=(a_1+a_2)v_1+(b_1+b_2)v_2$
    \item which is span{$v_1$,$v_2$}
\end{itemize}
\item closed under scalar multiplication
\begin{itemize}
    \item $u=av_1+bv_2$
    \item $c*u=c(av_1+bv_2)=(c*a)v_1+(c*b)v_2$
\end{itemize}
which is span{$v_1$,$v_2$}
\end{itemize}
\end{ex}
\begin{imp:defn}{Column Space}{} The column space of a matrix A, denoted col(A), is the set of all linear combinations of the columns of A. (col(a) is a subspace)
\end{imp:defn}
\begin{ex}
Let $A=
\begin{bmatrix}
   1 & -3 & -4\\
   -4 & 6 & -2\\
   -3 & 7 & 6\\
\end{bmatrix}
$ and $b= 
\begin{bmatrix}
   3 \\
   3 \\
   -4\\
\end{bmatrix}$\\
is b in Col(A)?\\
if b is a linear combination of the columns of A,
the Ax=b has a solution. We are asking whether [A|b] is consistant of not.\\
$
\begin{bmatrix}
   1 & -3 & -4 & | & 3\\
   -4 & 6 & -2 & | & 3\\
   -3 & 7 & 6 & | & -4\\
\end{bmatrix}
$ goes to $
\begin{bmatrix}
   1 & -3 & -4 & | & 3\\
   0 & -6 & -18 & | & 15\\
   0 & 0 & 0 & | & 0\\
\end{bmatrix}
$\\
No pivot in augmented column, so b is in col(A).\\
Note: b in col(A) for every b in $\mathbb{R}^m$\\
\begin{itemize}
    \item Ax=b has a solution for every b in $\mathbb{R}^m$
    \item The columns of A span $\mathbb{R}^m$
    \item A has a pivot in every row in REF
\end{itemize}
\end{ex}
\begin{imp:defn}{Null Space}{} The null space of matrics A, denoted null(A), is the set of all solutions to Ax=$\vec o$
\end{imp:defn}
\begin{note} Any solution set of Ax=$\vec o$ can be written in parametric form.\\
$x=x_2\begin{bmatrix}
0\\
2\\
1\\
\end{bmatrix} + x_3 \begin{bmatrix}
0\\
0\\
1\\
\end{bmatrix} =$span$ \{ \begin{bmatrix}
0\\
2\\
1\\
\end{bmatrix} \begin{bmatrix}
0\\
0\\
1\\
\end{bmatrix} \}$\\
so null(A) is a subspace.
\end{note}
\begin{imp:defn}{Basis for a Subspace}{} A basis for a subspace H is a linearly independent set that spans H.
\end{imp:defn}
\begin{ex}
Find a basis for null(A), if
$A=\begin{bmatrix}
-3 & 6 & -1 & 1 &-7\\
1 & -2 & 2 & 3 & -1\\
2 & -4 & 5 & 8 &-4 \\
\end{bmatrix}$\\
$A=\begin{bmatrix}
1 & -2 & 0 & -1 & -3 & | & 0\\
0 & 0 & 1 & 2 & -2 & | & 0\\
0 & 0 & 0 & 0 & 0  & | & 0\\
\end{bmatrix}$
\begin{align*}
    x_1-2x_2-x_4+3x_5&=0\\
    x_2&=x_2\\
    x_3+2x_4-2x_5&=0\\
    x_4&=x_4\\
    x_5&=x_5
\end{align*}

\end{ex}